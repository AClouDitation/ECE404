\documentclass[11pt]{article}

% ------
% LAYOUT
% ------
\textwidth 165mm %
\textheight 230mm %
\oddsidemargin 0mm %
\evensidemargin 0mm %
\topmargin -15mm %
\parindent= 10mm

\usepackage[dvips]{graphicx}
\usepackage{multirow,multicol}
\usepackage[table]{xcolor}

\usepackage{amssymb}
\usepackage{amsfonts}
\usepackage{amsthm}
\usepackage{amsmath}

\usepackage{subfigure}
\usepackage{minted}
\usepackage{listings}

\usepackage{scalerel}
\usepackage{stackengine}
\newcommand\showdiv[1]{\overline{\smash{\hstretch{.5}{)}\mkern-3.2mu\hstretch{.5}{)}}#1}}
\newcommand\ph[1]{\textcolor{white}{#1}}

\lstset{
  breaklines=true,
}

\begin{document}
\noindent Homework Number: 04\\
Name: Yi Qiao\\
ECN Login: qiao22\\
Due Date: 02/12/2019\\

\section*{Theory Problems}
\subsection*{1.}
\subsubsection*{(a)}
$$(2x+3)-(2x^3+3x-2)$$
\begin{equation}
\begin{split}
&=-2x^3-x+3+2\\
&=3x^3+4x
\end{split}
\end{equation}

\subsubsection*{(b)}
$$(3x^3+x-2)\times(x^2+4x+3)$$
\begin{equation}
\begin{split}
&=3x^5+12x^4+9x^3+x^3+4x^2+3x-2x^2-8x-6\\
&=3x^5+2x^4+2x^2+4
\end{split}
\end{equation}

\subsubsection*{(c)}
$$\frac{x^3+3x+4}{3x^2+2x+1}$$
\begin{center}
	\setstackgap{S}{1.5pt}
	\stackMath\def\stackalignment{r}
	\(
	\stackunder{%
		3x^2+2x+1 \stackon[1pt]{\showdiv{x^3\ph{+4x^2}\ +3x+4}}{2x+3\ \ \ \ \ \ \ \ \ \ }%
	}
	{%
		\Shortstack[l]{
			{\underline{x^3+4x^2+2x\ph{+4}\ }} 
			\ph{x^3+\ \ }4x^2+x+4 
			{\ph{x^3+\ \ }\underline{4x^2+x+3}}
			\ph{x^3+4x^2+x+\ \ }1
		}%
	}
	\)
\end{center}
thus,
\begin{equation}
	\frac{x^3+3x+4}{3x^2+2x+1}=2x+3+\frac{1}{3x^2+2x+1}
\end{equation}

\subsection*{2.}
\subsubsection*{(a)}
$$(x^3+x)\times(x+1)$$
\begin{equation}
\begin{split}
&=(x^4+x^2+x^3+x)\mod(x^4+x^3+1)\\
&\because \ x^4+x^3+1=0\\
&\therefore x^4=x^3+1\\
&=x^3+1+x^2+x^3+x\\
&=x^2+x+1
\end{split}
\end{equation}

\pagebreak
\subsubsection*{(b)}
$$(x^2+x)-(x^3+x^2+1)$$
\begin{equation}
\begin{split}
&=x^3+x+1
\end{split}
\end{equation}

\subsubsection*{(c)}
$$\frac{x^3+x^2+x+1}{x^2+1}$$
\begin{center}
	\setstackgap{S}{1.5pt}
	\stackMath\def\stackalignment{r}
	\(
	\stackunder{%
		x^2+1 \stackon[1pt]{\showdiv{x^3+x^2+x+1}}{x+1\ \ \ \ \ \ \ \ \ \ \ \ \ }%
	}
	{%
		\Shortstack[l]{
			{\underline{x^3\ \ \ \ \ \ +x\ \ \ \ \ }} 
			\ph{x^3+}x^2\ \ \ \ \ +1 {\ph{x^3+}\underline{x^2\ \ \ \ \ +1}}
			\ph{x^3+x^2\ \ \ \ \ +}0
		}%
	}
	\)
\end{center}
thus,
\begin{equation}
\frac{x^3+x^2+x+1}{x^2+1}=x+1
\end{equation}

\subsection*{3.}
\begin{equation}
\begin{split}
&\ \ 0\\
g^0&=1\\
g^1&=g\\
g^2&=g^2\\
g^3&=g^3\\
g^4=-g-1&=g+1\\
g^5=g(g^4)=g(g+1)&=g^2+g\\
g^6=g(g^5)=g(g^2+g)&=g^3+g^2\\
g^7=g(g^6)=g(g^3+g^2)=g^4+g^3&=g^3+g+1\\
g^8=g(g^7)=g(g^3+g+1)=g^4+g^2+g&=g^2+1\\
g^9=g(g^8)=g(g^2+1)&=g^3+g\\
g^{10}=g(g^9)=g(g^3+g)=g^4+g^2&=g^2+g+1\\
g^{11}=g(g^{10})=g(g^2+g+1)&=g^3+g^2+g\\
g^{12}=g(g^{11})=g(g^3+g^2+g)=g^4+g^3+g^2&=g^3+g^2+g+1\\
g^{13}=g(g^{12})=g(g^3+g^2+g+1)=g^4+g^3+g^2+g&=g^3+g^2+1\\
g^{14}=g(g^{13})=g(g^3+g^2+1)=g^4+g^3+g=g^3+1\\
\\
g^{15}=g(g^{14})=g(g^3+1)=g^4+g&=1\\
\vdots
\end{split}
\end{equation}
\end{document}