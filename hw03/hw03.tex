\documentclass[11pt]{article}

% ------
% LAYOUT
% ------
\textwidth 165mm %
\textheight 230mm %
\oddsidemargin 0mm %
\evensidemargin 0mm %
\topmargin -15mm %
\parindent= 10mm

\usepackage[dvips]{graphicx}
\usepackage{multirow,multicol}
\usepackage[table]{xcolor}

\usepackage{amssymb}
\usepackage{amsfonts}
\usepackage{amsthm}
\usepackage{amsmath}

\usepackage{subfigure}
\usepackage{minted}
\usepackage{listings}

\lstset{
  breaklines=true,
}

\begin{document}
\noindent Homework Number: 03\\
Name: Yi Qiao\\
ECN Login: qiao22\\
Due Date: 1/31/2019\\
\section*{Theory Problems}
\subsection*{1.} 
If the two operator is switched, it will no longer be a ring for the following reasons:\\
\indent1.Assuming the set is $\mathbb{R}$, multiplication does not have a identity element in the set.\\
\indent2.Even if we choose a better set, for example $(0,\infty]$, which have a identity element for multiplication, the ring operator,($+$ in this case) does not distribute over the group operator ($\times$).

\subsection*{2.}
	Euclid's Algorithm:

	\begin{equation}	
	\begin{aligned}[b]
	gcd(1344,752)&=gcd(752,1344\%752)&	&=gcd(752,592)\\
	&=gcd(592,752\%592)	&	&=gcd(592,160)\\
	&=gcd(160,592\%160)	&	&=gcd(160,112)\\
	&=gcd(112,160\%112)	&	&=gcd(112,48)\\
	&=gcd(48,112\%48)	&	&=gcd(48,16)\\
	&=gcd(16,48\%16)	&	&=gcd(16,0)\\
	&=16
	\end{aligned}
	\end{equation}
	
	Stein's Algorithm:
	\begin{equation}
	\begin{split}
	gcd(1344,752)&=2*gcd(672,376)\\
	&=4*gcd(336,188)\\
	&=8*gcd(168,94)\\
	&=16*gcd(84,47)\\
	&=16*gcd(42,47)\\
	&=16*gcd(21,47)\\
	&=16*gcd(13,21)\\
	&=16*gcd(4,13)\\
	&=16*gcd(2,13)\\
	&=16*gcd(1,13)\\
	&=16*gcd(6,1)\\
	&=16*gcd(3,1)\\
	&=16*gcd(1,1)\\
	&=16
	\end{split}
	\end{equation}

\pagebreak

\subsection*{3.}
Suppose there exist $\alpha=25^{-1}$ and a ring identity $1$ in $Z_{30}$
\begin{equation}
\begin{split}
(\alpha\times25)\ mod\ 30&= 1\\
\alpha\times25&=30\times n+1\ where\ n\in\mathbb{Z}\\
\alpha&=\frac{6}{5}\times n + \frac{1}{25} 
\end{split}
\end{equation}
By inspection, there does not exist such a $n\in\mathbb{Z}$ making $\alpha\in\mathbb{Z}$\\
Thus, $25$ does not have a multiplicative inverse in $Z_{30}$

\subsection*{4.}
\begin{equation}
\begin{aligned}[b]
gcd(33,23)	&=gcd(23,10)	&residue\ 10&= 1\times 33-1\times 23\\
		  	&=gcd(10,3)		&residue\  3&= 1\times 23-2\times 10\\
		  	&				&			&= 1\times 23-2\times (1\times 33-1\times 23)\\
		  	&				&			&= 3\times 23-2\times 33\\
		  	&=gcd(3,1)		&residue\  1&= 1\times 10-3\times 3\\
		  	&				&			&= 1\times(1\times 33-1\times 23)-3\times(3\times 23-2\times 33)\\
		  	&				&			&= 7\times 33-10\times 23
\end{aligned}
\end{equation}
Therefore, the inverse of $23$ in $Z_{33}$ is $23$
\subsection*{5.}
\subsubsection*{(a)}
$$8x\equiv 5(mod\ 23)$$
Since 8 is prime relative to 23, there exist a multiplicative inverse in $Z_{23}$ for $8$, which can be obtained by the Extended Euclid's algorithm:
\begin{equation}
\begin{aligned}[b]
gcd(23,8)	&=gcd(8,7)	&residue\ 7	&= 1\times 23-2\times 8\\
			&=gcd(7,1)	&residue\ 1	&= 1\times 8 -1\times 7\\
			&			&		   	&= 1\times 8 -1\times (1\times 23-2\times 8)\\
			&			&			&= 3\times 8 -1\times 23\\
\end{aligned}
\end{equation}
Thus, $8^{-1}$ in $Z_{23}$ is 3
\begin{equation}
\begin{split}
8x&\equiv 5(mod\ 23)\\
x&= 5\times 8^{-1}=(5\times 3)\ mod\ 23\\
&=15
\end{split}
\end{equation}
\pagebreak
\subsubsection*{(b)}
$$6x\equiv 3(mod\ 19)$$
For the same rational in (a)\\
\begin{equation}
\begin{aligned}[b]
gcd(19,6)	&=gcd(6,1)	&residue\ 1	&= 1\times 19-3\times 6\\
\end{aligned}
\end{equation}
Thus, $6^{-1}$ in $Z_{19}$ is $-3\ mod\ 19=16$
\begin{equation}
\begin{split}
6x&\equiv 3(mod\ 19)\\
x&=3\times 6^{-1}=3\times 16\ mod\ 19\\
&=10
\end{split}
\end{equation}
	
\subsubsection*{(c)}
$$25x\equiv 9(mod\ 7)$$
For the same rational in (a)\\
\begin{equation}
\begin{aligned}[b]
gcd(25,7)	&=gcd(7,4)	&residue\ 4	&= 1\times 25-3\times 7\\
			&=gcd(4,3)	&residue\ 3 &= 1\times 7-1\times 4\\
			&			&			&= 1\times 7-1\times (1\times 25-3\times 7)\\
			&			&			&= 4\times 7-1\times 25\\
			&=gcd(3,1)  &residue\ 1 &= 1\times 4-1\times 3\\
			&			&			&= (1\times 25-3\times 7)-(4\times 7 - 1\times 25)\\
			&			&			&= 2\times 25-7\times 7
\end{aligned}
\end{equation}
Thus, $25^{-1}=4^{-1}$ in $Z_{7}$ is $2$
\begin{equation}
\begin{split}
25x&\equiv 9(mod\ 7) = 2\\
x&=2\times 4^{-1}=2\times 2\ mod\ 7\\
&=4
\end{split}
\end{equation}
\pagebreak
\section*{Programming Question}
Since if $Z_n$ is a field, n has to be prime, otherwise, we can always found some integer $n\in[0..n-1]$
who does not have a multiplicative inverse in $Z_n$. So, the program is essentially judging the input is prime or not.
\subsection*{Code}  
\inputminted[breaklines]{python}{hw03.py}

\end{document}