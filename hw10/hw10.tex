\documentclass[11pt]{article}

% ------
% LAYOUT
% ------
\textwidth 165mm %
\textheight 230mm %
\oddsidemargin 0mm %
\evensidemargin 0mm %
\topmargin -15mm %
\parindent= 10mm

\usepackage[dvips]{graphicx}
\usepackage{multirow,multicol}
\usepackage[table]{xcolor}

\usepackage{amssymb}
\usepackage{amsfonts}
\usepackage{amsthm}
\usepackage{amsmath}

\usepackage{subfigure}
\usepackage{minted}
\usepackage{listings}
\usepackage{seqsplit}


\begin{document}
\noindent Homework Number: 10\\
Name: Yi Qiao\\
ECN Login: qiao22\\
Due Date: 04/04/2019\\

\section*{Buffer Overflow Attacks}
\subsection*{My buffer overflow string:}
AAAAAAAAAAAAAAAAAAAAAAAAAAAAA\textbackslash x3B\textbackslash x4F\textbackslash x55\textbackslash x55\textbackslash x55\textbackslash x55\textbackslash x00\textbackslash x00
\subsection*{Explanation}
The string consists of two parts. The first 29 "A"s and following the 8 bytes desired address. In the server's code, the size of the "str" buffer is 5 and anything beyond 5 characters will leads to a buffer overflow. So, we want the overflow overwrites the return address of the current call frame. From GDB, I get \&str at 0x7fffffffd6fb and (unsigned*) \$rbp at 0x7fffffffd710. Thus, the return address must live in 0x7fffffffd718. the difference between the return address and \&str is 29, thus there are 29 "A"s and follows the desired address in a byte-reversed fashion.

\section*{Code}
\inputminted[breaklines]{c}{new_server.c}
\end{document}